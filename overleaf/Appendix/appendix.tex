\chapter{Design drawings}
In this chapter, the extracts from the design drawings relevant to this Thesis are presented. The general notes on the materials and load combinations are given in Section \ref{app:general_notes}. The cross-sections of the different segments are presented in Section \ref{app:cross_sections}. The weight of the bridge is estimated from the summary of quantities in Section \ref{app:weight}. Ultimately, a summary of the final design verifications is given in Section \ref{app:design_verifications}. The values refer to the American units \SI{}{ft}, \SI{}{in}, \SI{}{kip}, \SI{}{kip ft} and \SI{}{ksi}. The general plan is shown in Fig. \ref{fig:general_plan}.
\begin{figure}[H]
    \centering
    \includegraphics[width=\textwidth]{overleaf/Appendix/Design drawings/General Plan.png}
    \caption{General plan}
    \label{fig:general_plan}
\end{figure}

\newpage
\section{General notes} \label{app:general_notes}
The general notes on the materials, the hanger resistance factors and the load combinations for cable loss and tie fracture are presented in Figs. \ref{fig:materials} to \ref{fig:hanger_load_combination}.
\begin{figure}[H]
    \centering
    \includegraphics[width=0.65\textwidth]{overleaf/Appendix/Design drawings/Material design data.png}
    \caption{Material design data}
    \label{fig:materials}
\end{figure}
\begin{figure}[H]
    \centering
    \includegraphics[trim={0 0.9cm 0 0cm},clip, width=0.6\textwidth]{overleaf/Appendix/Design drawings/Hanger resistance factor.PNG}
    \caption{Hanger resistance factors}
\end{figure}
\begin{figure}[H]
    \centering
    \includegraphics[trim={0 0.5cm 0 0.7cm},clip, width=0.69\textwidth]{overleaf/Appendix/Design drawings/Hanger load combination.png}
    \caption{Load combinations for extreme events and dynamic amplification factors}
    \label{fig:hanger_load_combination}
\end{figure}
%\begin{figure}[H]
%    \centering
%    \includegraphics[width=0.8\textwidth]{overleaf/Appendix/Design drawings/Tension tie.PNG}
%    \caption{Tie fracture load combination}
%    \label{fig:tie_fracture_loads}
%\end{figure}

\section{Cross-sections} \label{app:cross_sections}
In this section, the cross-sections are introduced. The dimensions and the resistances were taken from the design verification tables, which are shown in Section \ref{app:design_verifications}. The materials of the cross-sections are adapted from the design data specified in Figure \ref{fig:materials}.
The cross-sectional properties of the arch rib and the tie girder are shown in \cref{tab:cs_arch} and \cref{tab:cs_tie}.

\begin{table}[H]
\centering
\resizebox{0.85\textwidth}{!}{%
\begin{tabular}{lcccc}
\hline
Cross-Section          & Arch 1             & Arch 2             & Arch 3             &            \\ \hline
Flange width           & 1.219              & 1.219              & 1.219              & {[}m{]}    \\
Flange thickness       & 0.051              & 0.051              & 0.051              & {[}m{]}    \\
Subflange width        & 1.715              & 1.715              & 1.715              & {[}m{]}    \\
Subflange thickness    & 0.064              & 0.048              & 0.044              & {[}m{]}    \\
Web height             & 0.305              & 0.305              & 0.203              & {[}m{]}    \\
Web thickness          & 0.044              & 0.044              & 0.044              & {[}m{]}    \\ \hline
Area                   & 0.369              & 0.314              & 0.294              & {[}m$^2${]}   \\
Moment of intertia (z) & 0.150              & 0.137              & 0.134              & {[}m$^4${]}   \\
Moment of intertia (y) & 0.111              & 0.087              & 0.081              & {[}m$^4${]}   \\ \hline
Material               & Grade 270 HPS 70 W & Grade 270 HPS 70 W & Grade 270 HPS 50 W & {[}-{]}    \\
Yield strength         & 485                & 485                & 345                & {[}MPa{]}  \\
Young's modulus        & 210                & 210                & 210                & {[}GPa{]}  \\ \hline
Axial stiffness        & 77429              & 65997              & 61814              & {[}MN{]}   \\
Bending stiffness      & 31473              & 28673              & 28113              & {[}MNm$^2${]} \\
Weight                 & 2894.4             & 2467.0             & 2310.7             & {[}kg/m{]} \\ \hline
Normal resistance      & 130.0              & 108.8              & 82.3               & {[}MN{]}   \\
Moment resistance (z)  & 78.8               & 71.5               & 50.0               & {[}MNm{]}  \\
Moment resistance (y)  & 79.1               & 63.4               & 42.7               & {[}MNm{]}  \\ \hline
\end{tabular}
}
\caption{Cross-sectional properties of the arch rib segments}
\label{tab:cs_arch}
\end{table}
\begin{table}[H]
\centering
\resizebox{0.85\textwidth}{!}{%
\begin{tabular}{lcccc}
\hline
Cross-Section          & Tie 1              & Tie 2              & Tie 3 - 4          &            \\ \hline
Flange width           & 1.359              & 1.327              & 1.314              & {[}m{]}    \\
Flange thickness       & 0.051              & 0.044              & 0.044              & {[}m{]}    \\
Subflange width        & 0.152              & 0.203              & 0.197              & {[}m{]}    \\
Subflange thickness    & 0.025              & 0.025              & 0.025              & {[}m{]}    \\
Web height             & 2.134              & 2.083              & 2.083              & {[}m{]}    \\
Web thickness          & 0.064              & 0.038              & 0.029              & {[}m{]}    \\ \hline
Area                   & 0.425              & 0.297              & 0.256              & {[}m$^2${]}   \\
Moment of intertia (z) & 0.293              & 0.220              & 0.204              & {[}m$^4${]}   \\
Moment of intertia (y) & 0.1341             & 0.0860             & 0.0695             & {[}m$^4${]}   \\ \hline
Material               & Grade 270 HPS 70 W & Grade 270 HPS 70 W & Grade 270 HPS 70 W & {[}-{]}    \\
Yield strength         & 485                & 485                & 485                & {[}MPa{]}  \\
Young's modulus        & 210                & 210                & 210                & {[}GPa{]}  \\ \hline
Axial stiffness        & 89148              & 62441              & 53736              & {[}MN{]}   \\
Bending stiffness      & 61596              & 46255              & 42811              & {[}MNm$^2${]} \\
Weight                 & 3332.4             & 2334.1             & 2008.7             & {[}kg/m{]} \\ \hline
Normal resistance      & 153.2              & 117.1              & 100.5              & {[}MN{]}   \\
Moment resistance (z)  & 100.8              & 82.8               & 76.2               & {[}MNm{]}  \\
Moment resistance (y)  & 76.2               & 56.6               & 45.8               & {[}MNm{]}  \\ \hline
\end{tabular}
}
\caption{Cross-sectional properties of the tie girder segments}
\label{tab:cs_tie}

\end{table}

\newpage
The cross-sectional properties of the cables are shown in \cref{tab:cs_cable}.

\begin{table}[H]
\centering
\begin{tabular}{lcc}
\hline
Cross-Section          & Cable                                  &            \\ \hline
Strand diameter        & 15.2                                   & {[}mm{]}   \\
Strand area            & 140                                    & {[}mm$^2${]}  \\
Number of strands      & 29                                     & {[}-{]}    \\ \hline
Area                   & 0.0041                                 & {[}m$^2${]}   \\
Moment of intertia (z) & -                                      & {[}m$^4${]}   \\
Moment of intertia (y) & -                                      & {[}m$^4${]}   \\ \hline
Material               & Grade 270 Low Relaxation 7 Wire Strand & {[}-{]}    \\
Yield strength         & 1675                                   & {[}MPa{]}  \\
Young's modulus        & 196                                    & {[}GPa{]}  \\ \hline
Axial stiffness        & 796                                    & {[}MN{]}   \\
Bending stiffness      & -                                      & {[}MNm$^2${]} \\
Weight                 & 31.9                                   & {[}kg/m{]} \\ \hline
Normal resistance      & 6.8                                    & {[}MN{]}   \\
Moment resistance (z)  & -                                      & {[}MNm{]}  \\
Moment resistance (y)  & -                                      & {[}MNm{]}  \\ \hline
\end{tabular}

\caption{Cross-sectional properties of the cables}
\label{tab:cs_cable}
\end{table}

\section{Weights} \label{app:weight}
The weights of the structural and non-structural components are estimated from the bridge and steel quantities presented in Fig. \ref{fig:bridge_quantities} and \ref{fig:steel_quantities}. The weights of the arch and the tie is assumed as their steel weights, including half of the anchorage weights for each. Also the upper lateral bracing is assigned to the arch. The weight of the deck is estimated from the concrete quantity and as well as the lower lateral bracing and the stringers. The weight of the utilities is taken as \SI{9388}{kN} for the total weight to correspond to the unfactored vertical reaction specified in the pot bearing data. The distributed weights are derived in Table  \ref{tab:weights}.

\begin{figure}[H]
    \centering
    \includegraphics[trim={0 1.4cm 0 0},clip, width=0.7\textwidth]{overleaf/Appendix/Design drawings/estimated bridge quantities.PNG}
    \caption{Summary of estimated bridge quantities}
    \label{fig:bridge_quantities}
\end{figure}
\begin{figure}[H]
    \centering
    \includegraphics[trim={0 0cm 0 0cm},clip, width=0.65\textwidth]{overleaf/Appendix/Design drawings/Steel superstructure.PNG}
    \caption{Estimated steel quantities}
    \label{fig:steel_quantities}
\end{figure}

\begin{table}[H]
\centering
\caption{Estimated weights of the components}
\label{tab:weights}
%\resizebox{0.6\columnwidth}{!}
\end{table}

\section{Design verifications} \label{app:design_verifications}
The design verifications of the arch rib, the tie girder and the hangers are presented in Figs. \ref{fig:arch_design_verification} to \ref{fig:hanger_design_verification}.
\begin{figure}[H]
    \centering
    \includegraphics[width=\textwidth]{overleaf/Appendix/Design drawings/arch rib verifications.PNG}
    \caption{Design verifications of the arch rib segments}
    \label{fig:arch_design_verification}
\end{figure}
\begin{figure}[H]
    \centering
    \includegraphics[width=\textwidth]{overleaf/Appendix/Design drawings/tie girder verifications.PNG}
    \caption{Design verifications of the tie girder segments}
    \label{fig:tie_design_verification}
\end{figure}
\begin{figure}[H]
    \centering
    \includegraphics[width=\textwidth]{overleaf/Appendix/Design drawings/cable verifications.PNG}
    \caption{Design verifications of the hangers}
    \label{fig:hanger_design_verification}
\end{figure}



\newpage
\chapter{Model}
\label{AppendixModel}
\section{Hangers}
\label{Appendix_A_Hangers}
In this section, the non-linear effects of the hangers are considered. Assuming a parabolic displacement curve, the secondary effects are approximated by Eq. \eqref{eq:cable}, which gives the cable elongation $\delta$ at a stress of $\sigma_2$. Where $a$ is the horizontal component of the cable length $c$ at a stress of $\sigma_1$, $E$ is the modulus of elasticity and $\gamma$ is the weight of the cable \cite{NIELS}. 

\begin{equation}
    \frac{\delta}{c}= \frac{\sigma_2-\sigma_1}{E} 
    + \frac{\gamma^2\,c^2}{24}\,\left(\frac{1}{\sigma_1^2} -\frac{1}{\sigma_2^2} \right) 
    + \frac{\gamma^2\,c^2}{12\,E}\,\left(\frac{1}{\sigma_2} -\frac{1}{\sigma_1} \right)
    \label{eq:cable}
\end{equation}

To judge the influence on the Blennerhassett Island Bridge, the following values are assumed: $a=\SI{26}{m}$, $c=\SI{59.5}{m}$, $E=\SI{196}{GPa}$, $\gamma=\SI{78.5}{kN/m^3}$ and $\sigma_1=0.45\,f_y=\SI{750}{MPa}$. The elongation of the cable is compared to the assumed linear elastic behaviour of a rod, which is shown in Fig. \ref{fig:hanger_approximation}.

\begin{figure}[H]
    \centering
    \includegraphics[width=0.7\textwidth]{overleaf/Appendix/Pictures/hanger_approximation.png}
    \caption{Cable elongation considering secondary effects}
    \label{fig:hanger_approximation}
\end{figure}

Only for cables stresses below \SI{100}{MPa}, which corresponds to 6\% of its yield stress, secondary effects become relevant. As stresses in this range are not occurring in any calculation, secondary effects can be neglected.

\newpage
\section{Live loading} \label{Appendix_Liveloading}
In this section, the arrangement of the live loads on the deck is investigated. The three design states strength, cable replacement and cable loss are treated individually as different circumstances apply. The live load relevant for fatigue can then be derived from the strength limit state. In the first step, the location of each lane is defined. According to the AASHTO design provisions, the width of every lane is equal to $\SI{12}{ft} = \SI{3.7}{m}$ of which \SI{3}{m} are loaded and the center of the applied forces $x_c$ can be derived \cite{AASHTO}. The force applied to the tie girder $F_r$ can then be calculated according to Eq. \ref{eq:reaction} using the width of the deck $w_{deck}=\SI{107}{ft}=\SI{32.6}{m}$.
\begin{equation}
    F_r = \frac{x_c}{w_{deck}}
    \label{eq:reaction}
\end{equation}
In a second step, the amount of loaded lanes under the consideration of the multiple presence factor (MPF) is determined. As the multiple presence factor decreases with the number of loaded lanes, each possibility is calculated to find the worst arrangement. The calculations are conducted for a unit lane load. The obtained factor relates the load on each lane to the load applied to the investigated tie girder. The factor can then be used for the calculation of the force on the tie of the distributed lane load ($q_{LL}=\SI{9.3}{kN/m}$) as well as of the design truck load ($Q_{LL}=\SI{325}{kN}$). For the design truck load, an additional dynamic multiplier of 1.33 is taken into account. 

\subsection*{Vehicular use} \label{Appendx_A_Live_loading_1}
In the first strength limit state, the entire width of the deck is available to traffic. The lanes are arranged as densely to one side as possible with the first lane starting at $\SI{4.6}{ft}=\SI{1.4}{m}$ from the investigated arch plane. The calculations presented in Table \ref{tab:app_ll_uls} show that the worst arrangement results with six loaded lanes. 

\begin{table}[H]
\centering
\begin{tabular}{cccccccccccc}
\cline{2-11}
             & Lane     &  & 1    & 2    & 3    & 4    & 5    & 6    & 7    & 8    &      \\
             & Centroid &  & 2.9  & 6.6  & 10.2 & 13.9 & 17.5 & 21.2 & 24.8 & 28.5 &      \\
             & Reaction &  & 0.91 & 0.80 & 0.69 & 0.57 & 0.46 & 0.35 & 0.24 & 0.13 &      \\ \hline
Loaded Lanes & MPF      &  &      &      &      &      &      &      &      &      & Sum  \\ \hline
1            & 1.2      &  & 1.09 &      &      &      &      &      &      &      & 1.09 \\
2            & 1.0      &  & 0.91 & 0.80 &      &      &      &      &      &      & 1.71 \\
3            & 0.85     &  & 0.77 & 0.68 & 0.58 &      &      &      &      &      & 2.04 \\
4            & 0.75     &  & 0.68 & 0.60 & 0.52 & 0.43 &      &      &      &      & 2.23 \\
5            & 0.70     &  & 0.64 & 0.56 & 0.48 & 0.40 & 0.32 &      &      &      & 2.40 \\
6            & 0.65     &  & 0.59 & 0.52 & 0.45 & 0.37 & 0.30 & 0.23 &      &      & 2.46 \\
7            & 0.60     &  & 0.55 & 0.48 & 0.41 & 0.34 & 0.28 & 0.21 & 0.14 &      & 2.41 \\
8            & 0.55     &  & 0.50 & 0.44 & 0.38 & 0.32 & 0.25 & 0.19 & 0.13 & 0.07 & 2.28 \\ \hline
\end{tabular}
\caption{Force on tie girder for ultimate limit state under unit lane load}
\label{tab:app_ll_uls}
\end{table}

Six loaded lanes yield a factor of 2.46. Hence the load applied on the investigated arch plane can be calculated according to Eqs. \eqref{eq:q_ll_uls} and \eqref{eq:Q_ll_uls}. The decisive arrangement is illustrated in Fig. \ref{fig:app_hangers_uls}.
\begin{equation}
    q_{ll, uls} = 2.46 \cdot \SI{9.3}{kN/m} = \SI{22.9}{kN/m}
    \label{eq:q_ll_uls}
\end{equation}
\begin{equation}
    Q_{ll, uls} = 2.46 \cdot 1.33 \cdot \SI{325}{kN} = \SI{1063}{kN}
    \label{eq:Q_ll_uls}
\end{equation}

\begin{figure}[H]
    \centering
    \includegraphics[width=\textwidth]{overleaf/Appendix/Pictures/Cross_Section_LL_ULS.PNG}
    \caption{Live load arrangement in the strength limit state}
    \label{fig:app_hangers_uls}
\end{figure}


\subsection*{Fatigue} \label{Appendx_A_Live_loading_15}
The fatigue limit state is investigated according to the recommendations for stay cable design, testing and installation by the Post Tensioning Institute (PTI). The corresponding loading is composed of a single design truck which occupies the lane resulting in the highest effect on the investigated stay cable \cite{PTI}. It corresponds to the lane closest to the border in this case, for which, according to Table \ref{tab:app_ll_uls}, the force on the tie girder is equal to 91\% of the design truck load. It results in the fatigue load according to Eq. \ref{eq:Q_fat}.

\begin{equation}
    Q_{fat} = 0.91 \cdot \SI{325}{kN} = \SI{296}{kN}
    \label{eq:Q_fat}
\end{equation}

\subsection*{Cable replacement} \label{Appendx_A_Live_loading_2}
In the event of cable replacement, one lane is shifted away from the hanger being exchanged. Besides that, the lanes are located at the same positions as in the ultimate limit state. A load for the replacement truck is disregarded for in this investigation. The calculation of the different arrangements is presented in Table \ref{tab:app_ll_hanger_replacement}.
\begin{table}[H]
\centering
\caption{Force on tie girder for hanger replacement under unit lane load}
\label{tab:app_ll_hanger_replacement}
\begin{tabular}{cccccccccccc}
\cline{2-11}
             & Lane     &  & 1    & 2    & 3    & 4    & 5    & 6    & 7    & 8    &      \\
             & Centroid &  & 2.9  & 6.6  & 10.2 & 13.9 & 17.5 & 21.2 & 24.8 & 28.5 &      \\
             & Reaction &  & 0.91 & 0.80 & 0.69 & 0.57 & 0.46 & 0.35 & 0.24 & 0.13 &      \\ \hline
Loaded Lanes & MPF      &  &      &      &      &      &      &      &      &      & Sum  \\ \hline
1            & 1.2      &  & -    & 0.96 &      &      &      &      &      &      & 0.96 \\
2            & 1.0      &  & -    & 0.80 & 0.69 &      &      &      &      &      & 1.49 \\
3            & 0.85     &  & -    & 0.68 & 0.58 & 0.49 &      &      &      &      & 1.75 \\
4            & 0.75     &  & -    & 0.60 & 0.52 & 0.43 & 0.35 &      &      &      & 1.89 \\
5            & 0.70     &  & -    & 0.56 & 0.48 & 0.40 & 0.32 & 0.25 &      &      & 2.01 \\
6            & 0.65     &  & -    & 0.52 & 0.45 & 0.37 & 0.30 & 0.23 & 0.15 &      & 2.02 \\
7            & 0.60     &  & -    & 0.48 & 0.41 & 0.34 & 0.28 & 0.21 & 0.14 & 0.08 & 1.94 \\ \hline
\end{tabular}
\end{table}

The resulting loads on the investigated arch plane are presented in Eqs. \eqref{eq:q_ll_repl} and \eqref{eq:Q_ll_repl}.
\begin{equation}
    q_{ll, repl} = 2.02 \cdot \SI{9.3}{kN/m} = \SI{18.8}{kN/m}
    \label{eq:q_ll_repl}
\end{equation}
\begin{equation}
    Q_{ll, repl} = 2.02 \cdot 1.33 \cdot \SI{325}{kN} = \SI{874}{kN}
    \label{eq:Q_ll_repl}
\end{equation}

\subsection*{Cable loss} \label{Appendx_A_Live_loading_3}
In the extreme event of cable loss, not the deck's entire width is available to the live loads. For this case, only the four actual marked lanes are available. Their locations were taken from the design drawings. The investigation in Table \ref{tab:app_ll_cable_loss} shows that all of these lanes are loaded in the worst arrangement.
\begin{table}[H]
\centering
\begin{tabular}{cccccccc}
\cline{2-7}
             & Lane     &  & 1    & 2    & 3    & 4    &      \\
             & Centroid &  & 8.4  & 12.0 & 20.0 & 23.6 &      \\
             & Reaction &  & 0.74 & 0.63 & 0.39 & 0.28 &      \\ \hline
Loaded Lanes & MPF      &  &      &      &      &      & Sum  \\ \hline
1            & 1.2      &  & 0.89 &      &      &      & 0.89 \\
2            & 1.0      &  & 0.74 & 0.63 &      &      & 1.37 \\
3            & 0.85     &  & 0.63 & 0.54 & 0.33 &      & 1.50 \\
4            & 0.75     &  & 0.56 & 0.47 & 0.29 & 0.21 & 1.53 \\ \hline
\end{tabular}
\caption{Force on tie girder for cable loss under unit lane load}
\label{tab:app_ll_cable_loss}
\end{table}

The loads resulting on the tie girder are calculated in Eqs. \eqref{eq:q_ll_loss} and \eqref{eq:Q_ll_loss}.
\begin{equation}
    q_{ll, loss} = 1.53 \cdot \SI{9.3}{kN/m} = \SI{14.2}{kN/m}
    \label{eq:q_ll_loss}
\end{equation}
\begin{equation}
    Q_{ll, loss} = 1.42 \cdot 1.33 \cdot \SI{325}{kN} = \SI{660}{kN}
    \label{eq:Q_ll_loss}
\end{equation}

An illustration of the load arrangement for the event of cable loss is presented in Fig. \ref{fig:app_hangers_cable_loss}.

\begin{figure}[H]
    \centering
    \includegraphics[width=\textwidth]{overleaf/Appendix/Pictures/Cross_Section_LL_Cable Loss.PNG}
    \caption{Live load arrangement for cable loss}
    \label{fig:app_hangers_cable_loss}
\end{figure}

\newpage
\section{Verification} \label{app:model_verification}
The model is verified by comparing it to the model of a parallel Master Thesis by Moritz Studer. For the comparison, the elastic response under the dead loading is compared. The two responses are shown in Figures \ref{fig:verification_1} and \ref{fig:verification_2} respsectively.

\begin{figure}[H]
    \centering
    \includegraphics[trim={12cm 5cm 10cm 5cm},clip, width=\textwidth]{calculations/model comparison/dead_load.png}
    \caption{Elastic response for dead loading in the model of this Thesis}
    \label{fig:verification_1}
\end{figure}

\begin{figure}[H]
    \centering
    \includegraphics[trim={0.3cm 4cm 0.8cm 3.5cm},clip, width=\textwidth]{calculations/model comparison/dead_load_comparison.PNG}
    \caption{Elastic response for dead loading in the model of a parallel Thesis}
    \label{fig:verification_2}
\end{figure}

The responses in the two models qualitatively agree well. Both of them indicate the strongest moments in the arch and the tie near the knuckle. The respective moments of in the arch are \SI{7618}{kNm} in the used model and \SI{6993}{kNm} in the comparative model. For the tie, the difference is slightly larger at \SI{8959}{kNm} and \SI{7738}{kNm}. The generally larger moment magnitudes in the model of this Thesis can be explained by the assumption of higher dead loads. However, as the distributions visually agree very well, the model is considered to be verified.


\chapter{Cost estimation} \label{app:cost}
In this chapter, the hanger unit costs are derived. The unit costs for the free length, the detailing and installation and the anchorages were received from a partner in the industry for \SI{150}{mm^2} strands and are given in Table \ref{tab:hanger_unit}.

\begin{table}[H]
\centering
\begin{tabular}{ccc}
\hline
Free length      & Detailing and installation & Anchorages    \\
3.00 \$/strand/m & 12.5 \$/strand/m           & 185 \$/strand \\ \hline
\end{tabular}
\caption{Hanger unit costs}
\label{tab:hanger_unit}
\end{table}

The total costs are calculated in Table \ref{tab:hanger_costs} for a single hanger set of the Blennerhassett Island Bridge. 

\begin{table}[H]
\centering
\resizebox{\textwidth}{!}{%
\begin{tabular}{lcccccc}
\hline
Hanger & Length  & Number of strands & Free length & Anchorage & Detailing and installation & Total cost \\
       & {[}m{]} & {[}-{]}           & {[}\${]}         & {[}\${]}       & {[}\${]}                   & {[}\${]}   \\ \hline
T1-H1  & 10.8    & 29                & \SI{938}{}              & \SI{5365}{}           & \SI{3907}{}                       & \SI{10210}{}      \\
T2-H2  & 21.5    & 29                & \SI{1868}{}             & \SI{5365}{}           & \SI{7784}{}                       & \SI{15017}{}      \\
T1-H3  & 24.1    & 29                & \SI{2097}{}             & \SI{5365}{}           & \SI{8738}{}                       & \SI{16201}{}      \\
T3-H4  & 31.3    & 29                & \SI{2721}{}             & \SI{5365}{}           & \SI{11339}{}                      & \SI{19425}{}      \\
T2-H5  & 38.3    & 29                & \SI{3332}{}             & \SI{5365}{}           & \SI{13882}{}                      & \SI{22579}{}      \\
T4-H6  & 39.8    & 29                & \SI{3461}{}             & \SI{5365}{}           & \SI{14421}{}                      & \SI{23247}{}      \\
TS-H7  & 46.9    & 29                & \SI{4083}{}             & \SI{5365}{}           & \SI{17012}{}                      & \SI{26460}{}      \\
T3-H8  & 48.8    & 29                & \SI{4245}{}             & \SI{5365}{}           & \SI{17686}{}                      & \SI{27295}{}      \\
T6-H9  & 52.6    & 29                & \SI{4574}{}             & \SI{5365}{}           & \SI{19059}{}                      & \SI{28998}{}      \\
T4-H10 & 55.1    & 29                & \SI{4797}{}             & \SI{5365}{}           & \SI{19986}{}                      & \SI{30147}{}      \\
T7-H11 & 56.5    & 29                & \SI{4917}{}             & \SI{5365}{}           & \SI{20488}{}                      & \SI{30770}{}      \\
TS-H12 & 58.2    & 29                & \SI{5061}{}             & \SI{5365}{}           & \SI{21087}{}                      & \SI{31512}{}      \\
T8-H13 & 58.5    & 29                & \SI{5088}{}             & \SI{5365}{}           & \SI{21201}{}                      & \SI{31654}{}      \\ \hline
Total  & 542.3   &                   & \SI{47181}{}            & \SI{69745}{}          & \SI{196588}{}                     & \SI{313515}{}     \\ \hline
\end{tabular}
}
\caption{Cost analysis for a hanger set of the Blennerhassett Island Bridge}
\label{tab:hanger_costs}
\end{table}

The costs per kilogram are determined in Eq. \ref{eq:hanger_unit_cost}, considering the hanger weight of \SI{31.9}{kg/m} and the area of \SI{140}{mm^2} per strand used in the Blennerhassett Island Bridge.

\begin{equation}
    \frac{\SI{313515}{\$}}{\SI{542.3}{m} \cdot \SI{31.9}{kg/m}} \, \frac{\SI{140}{mm^2}}{\SI{150}{mm^2}}= \SI{16.9}{\$/kg}
    \label{eq:hanger_unit_cost}
\end{equation}

\newpage
\chapter{Implementation}
All calculations are implemented using the Python programming language, which was chosen for its simplicity, code readability and its wide range of useful libraries. Besides other programming paradigms it supports object oriented programming, which uses objects as the basic building blocks of software. Objects structure the code by representing real-life entities and containing the corresponding data and functions. A class is a blueprint for a specific object, defining its properties and functions, which can interact with other classes. This blueprint may be used for the fast definition of multiple objects or of extended sub-classes. Generally, this paradigm allows the simple extension or reuse of existing code. Instead of explaining how the results of this Thesis were obtained, an introduction to the classes of the code are given in this chapter. The core functions for the structural analysis were adapted from an earlier project and are not explained in this chapter. For examination of the calculations or for reuse of the programmed code fragments the entire package can be found on \href{https://github.com/uridmo/networkarchoptimization}{GitHub}.

\subsection*{Bridge}
\begin{itemize}
    \item {\bf Description:} The bridge class allows the definition, calculation and investigation of a network tied-arch bridge. It is defined by a set of 20 characteristic parameters such as the arch, the distributed live loading, the cross sections, etc. It contains the structural model which is set up in the first part and calculated under all relevant load cases in the second part of its definition. Ultimately, the costs of the relevant cross sections are estimated. Further, it contains functions to display its structure or internal force effects.
    \item {\bf Properties:} span, rise, amount of floor beams, arch shape, arch optimisation method, curve fitting method, self stress state method and parameters, arch cross sections, tie cross sections, hanger cross section, amount of hangers, hanger arrangement and parameters, knuckle, deck weight, utilities weight, live loads, fatigue loads, extreme events and unit weight and cost of the anchorages.
    \item {\bf Functions:} (the calculation of the model is conducted in the initiation of the object), plot structure, plot effects, plot effects on structure, create overview tables of costs/dc ratio/internal forces, evaluate the cost function
\end{itemize}

\subsection*{Blennerhassett Bridge}
\begin{itemize}
    \item {\bf Description:} It is a sub-class of the Bridge class and allows the specific definition of models representing the Blennerhassett Island Bridge. The default settings are according to the final design and can be changed individually in the initiation. It defines the cross sections and passes them to the Bridge class.
    \item {\bf Properties:} same as Bridge class
    \item {\bf Functions:} same as Bridge class
\end{itemize}

\subsection*{Network Arch}
\begin{itemize}
    \item {\bf Description:} It represents the structural model of a Bridge object. It is defined and composed of the arch, the tie, the hangers and a set of nodes. The different load cases are calculated by invoking the respective function with the corresponding parameters. It also features the calculations of the extreme events and the respective assignment of internal force effects.
    \item {\bf Properties:} arch, tie, hangers, nodes
    \item {\bf Functions:} create model, set effects, get effects, set effect range, calculate load cases, assign wind effects, calculate strength limit state, calculate tie fracture, calculate cable loss
\end{itemize}

\subsection*{Nodes}
\begin{itemize}
    \item {\bf Description:} It is a collection of the nodes of a specific model. It allows adding nodes to the set while maintaining a specified minimum spacing between them. If a node is added too close to an existing one, no new node is created and the existing one is returned instead. Thereby, the stiffness matrix remains well determined as the structural element retain a minimum length. Each node has x- and y-components and an ordered index.
    \item {\bf Properties:} nodes list, amount, minimum spacing
    \item {\bf Functions:} add node, order node, get structural nodes
\end{itemize}

\subsection*{Element}
\begin{itemize}
    \item {\bf Description:}  It describes the individual components of the structural model, namely the arch, the tie and the hangers, rely on this class which saves and orders the effects for the load cases and limit states. Further this class gives functions for adding, multiplying and merging effects and effect ranges.
    \item {\bf Properties:} effects
    \item {\bf Functions:} set effects, get effects, get range, merge ranges, add ranges
\end{itemize}

\subsection*{Line Element}
\begin{itemize}
    \item {\bf Description:} It expands the element class for the components of the arch rib and the tie girder. It contains the ordered nodes of each component and the assigned cross sections between them. 
    \item {\bf Properties:} same as Element, nodes, cross sections
    \item {\bf Functions:} self weight, permanent loads, distributed loads, concentrated loads, get coordinates, plot elements, plot effects, assign permanent effects
\end{itemize}

\subsection*{Cross Sections}
\begin{itemize}
    \item {\bf Description:} This class contains the information related to the structural properties, the internal force effects and the cost estimation of each cross section. After the internal forces of the limit states are defined, they are assigned to the respective cross section which determines the maximum degree of compliance. The effects under wind loading are assigned to the respective cross section directly. This allows the estimation of the costs. Further, it calculates the stress in a fractured cross section.
    \item {\bf Properties:} name, weight, stiffness, unit costs, unit weight, length, cost, effects, degrees of compliance, resistances, wind effects, tie fractures
    \item {\bf Functions:} get self weight, get extreme internal forces, assign wind effects, calculate maximum d/c ratio, calculate tie fracture stress, calculate cost
\end{itemize}

\subsection*{Arch}
\begin{itemize}
    \item {\bf Description:} An extension of the Line Element class specifically for the arch rib. Additionally it allows the determination of the connection nodes of the hangers and the calculation of an optimal permanent tie tension force. It features sub-classes for the definition of the different arch shapes.
    \item {\bf Properties:} same as Line Element class, rise span, tie tension
    \item {\bf Functions:} connect nodes, assign tension by crown moment, assign tension by least squares
\end{itemize}

\subsection*{Tie}
\begin{itemize}
    \item {\bf Description:} Is is the extension of the line element class for the tie girder. It gives the loads for the different load cases and calculates the respective stresses for a fractured tie. Further it contains the elaborate methods for the definition of the self-equilibrium stress state. 
    \item {\bf Properties:} same as Line Element class, span, amount of floor beams, floor beam nodes, deck weight, utilities weight
    \item {\bf Functions:} assign hangers, self weight, non-structural loading, zero-displacement method, calculate fracture stresses
\end{itemize}

\subsection*{Hangers}
\begin{itemize}
    \item {\bf Description:} This class extends the element class for the specific use for the hanger component of the structural model. The hangers and also the knuckles are grouped into hanger sets which allow the addition of hangers or the plotting of the internal force effects. Each hanger is defined by the connection nodes, inclination, prestressing force and cross-section. Further, there are sub-classes for creating specific hanger arrangements.
    \item {\bf Properties:} same as Element class, hanger sets, hangers
    \item {\bf Functions:} get connection nodes, set effects, set prestressing forces, define knuckles, plot elements, plot effects
\end{itemize}




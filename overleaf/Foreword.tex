\section*{Foreword}\addcontentsline{toc}{section}{Foreword}

During an internship in the summer of 2018, I worked on a small pedestrian passage over a highway. It was the first new construction I was assigned to, and it was an exciting Vierendeel girder. As time was pressing and my team had substantial autonomy on the design, many decisions were made quickly and based on personal preference. Towards the end of the deadline, it seemed like we were just in time. Only then, we noticed that a specific serviceability verification could barely not be fulfilled. The project manager gave me the job to identify a simple change of the geometry or the cross-sections to solve the issue. But it was a challenging task and everything I tried backfired in one way or another. Neither any of the experienced engineers found the solution we were looking for. To solve the issue once and for all, it was decided to arrange diagonal high strength steel cables in each frame. All conditions were met, and the plans were sent out shortly afterwards. But it left me unsatisfied, knowing the cables were not necessary but also feeling poorly equipped to solve the issue.\medskip

The design of a network tied-arch bridge poses many more considerable challenges: Its high degree of intrinsic static indeterminacy gives the engineer control of the flow of forces, but it also complicates understanding its structural behaviour. As the bridge type makes advantageous use of the utilised materials, extreme events in which a component is lost can become decisive. Further, the entire geometry of the structure, including the hanger arrangement and the arch shape, is negotiable. This opens up a large space of plausible solutions and makes the civil engineer's decisions have a far-reaching influence on the efficiency and the financial expenses of the structure. These significant challenges are probably responsible for its sparse use, despite its aesthetic appeal and structural efficiency. \medskip

The next decades pose tremendous challenges in many scientific fields, and despite not often being in the spotlight, civil engineering also faces tall tasks. However, It seems that the method of trial and error relying on fundamental structural understanding and a handful of experience is considered sufficient for civil engineers. This feeling was accentuated by the fact that a lecture on structural optimisation was only offered at the department of mechanical engineering. However, for complicated and insufficiently investigated structures, such as a network tied-arch bridge, optimisation methods are an invaluable tool to overcome its challenges. I want to thank my supervisors, George Klonaris and Prof. Dr. Kaufmann, for giving me the opportunity to investigate and develop optimisation procedures for this particularly interesting bridge type. Besides finding optimisation potential for network tied-arch bridges, I hope to contribute to the use of more advanced methods in civil engineering. A diversified toolbox does not just facilitate more efficient designs, it also increases the enjoyment of the challenge itself.
\chapter{Conclusion}\label{sec:conclusion}
In this chapter, the obtained results are concluded into a few practical design guidelines. It should be considered that all results are derived for the circumstances of the Blennerhassett Island Bridge. Its most critical feature is the floating deck system with a floor beam and hanger spacing of almost \SI{20}{m}, an unmatched value by any other network tied-arch bridge. Further, the live-to-dead load ratio of $p/g=1/7$ is characteristic for road bridges with heavy concrete decks. Overall, the Blennerhassett Island Bridge can be considered an efficient structure and only little potential for optimisation is found in this investigation. However, the developed methods for the derivation of the arch shape and the self-equilibrium stress state constitute generally applicable tools for the initial design of network tied-arch bridges. They might be adapted for other circumstances rendering more potential for optimisation. Nevertheless, considering the investigated aspects, the following design guidelines are recommended:

\begin{itemize}
    \item The common parabolic and circular arch shapes are only suitable for vertical or perpendicular hanger arrangements. For the general arrangements, a detailed investigation facilitates the derivation of an efficient self-equilibrium stress state featuring uniform hanger forces.
    \item For the usual constant change of inclination arrangements, a quartic polynomial approximation suits the thrust line and can be predicted by the hypothetical case of a continuous hanger arrangement.
    \item As the thrust line is approximately linear between the hanger anchorage nodes on the arch, inevitable shape deviation moments occur for the permanent state. By arranging the hanger nodes more uniformly on the arch, the corresponding demand is reduced by a few per cents.
    \item For sparse hanger arrangements, it is recommended to adapt the arch shape to the thrust line under permanent loads. A continuous curvature causes large unnecessary bending moments.
    \item An efficient self-equilibrium stress state is obtained non-iteratively by considering the permanent supernumerary forces. The objective of reducing the maximum absolute bending moments in the arch rib and the tie girder can be formulated in an efficiently solvable linear programming problem.
    \item It is critical that the hangers are connected to the tie girder at the location of the floor beams. Otherwise, no efficient self-equilibrium stress state can be obtained.    Therefore, the radial arrangement is generally not recommended.
    \item An optimal hanger arrangement features uniform spacing on the arch rib and the tie girder.
    \item The structural behaviour only undergoes minor changes by an increased hanger and floor beam spacing. Only the extreme event of cable loss improves significantly. Therefore, it is recommended to optimise the deck system separately in a first step. The obtained floor beam density should only be increased if the demand for the cable loss extreme event governs the design.
    \item The hanger forces contribute the least to the estimated structural costs. However, they undergo the largest changes in costs with respect to the hanger arrangement.
    \item Steeper hangers reduce the dynamic demand in the arch for the extreme event of cable loss. On the other hand, flat inclinations allow for a reduction of the normal forces in the arch's knuckle area and the tie girder, thereby reducing the demand for the event of tie fracture.
    \item An optimal parallel hanger arrangement lies in the middle between $65\degree$ and $75\degree$.
    \item Steep hangers in the knuckle area improve the coupling of the arch and the tie and result in a more efficient elastic flow of forces for dead loads. It is facilitated most efficiently by the constant change of inclination arrangement featuring large changes between the hanger inclinations.
\end{itemize}


\cleardoublepage
\chapter{Outlook}\label{sec:outlook}
The complex structural system of network tied-arch bridges has been used for more than half a century. Its low weight and material use make it an up-and-coming option for medium spans in the future. Nevertheless, many aspects have not been investigated to adequately understand its structural behaviour and facilitate a simple and efficient initial design methodology. Its many design variables make its investigation and design particularly interesting but also challenging. This Thesis also only considered for a few relevant aspects. To allow for a further increase in popularity in the future, integral investigations of the critical aspects and assumptions are necessary. Also, some possible adaptations to the newly introduced methods are proposed:
\begin{itemize}
    \item A detailed comparison of 3-dimensional models to the simplified single arch plane models for both deck systems needs to be conducted to build the foundation for their consistent investigation.
    \item To start at the source of the flow of forces, a comparison of the different deck systems could give critical insight into the characteristic problems expected in the design.
    \item An aesthetic hanger arrangement pattern featuring relatively uniform hanger spacing on the arch rib and the tie girder could reduce the shape deviation moments while allowing for an efficient self-equilibrium stress state.
    \item Other materials than steel can facilitate some of the key challenges in the design and optimise the costs. In particular, the study of a concrete arch rib or carbon fibre hangers seems promising.
    \item The consideration of non-uniform hanger areas could improve its material use, as not all hangers are stressed equally.
    \item Analytical descriptions of the optimal arch shapes could facilitate their usability in the design process. Also, they could be adapted to account for more elaborate weight distributions.
    \item The arch and tie moment optimisation can be adapted to minimise the sum of each component's maximum moment. Thereby the self-equilibrium stress state might be slightly improved and applied to either optimised or standard arch shapes.
\end{itemize}

There are many more design variables to consider such as the arch bracing, its inclination and the rise-to-span ratio. For some of these aspects, investigations are conducted in parallel Master Theses at the institute of structural engineering at ETH Zurich.

